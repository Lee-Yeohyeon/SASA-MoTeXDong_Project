% Encoding: UTF-8


%%%%%%%%%%%%%%%%%%%%%%%%%%%%%% 
%참고 문헌 정리: biblatex을 이용하여, 국문 참고문헌, 영문 참고 문헌이 나오도록 만들었다. 참고문헌 목록을 만드는 곳에 \printbibliography[keyword=kobib] 와\printbibliography[notkeyword=kobib,heading=none] 를 써준다.
%%%%%%%%%%%%%%%%%%%%%%%%%%%%%%%5%
\usepackage[normalem]{ulem}
\usepackage[backend=biber,natbib=true,sorting=none,citestyle=numeric-comp,bibstyle=numeric,maxbibnames=10,minbibnames=6,maxcitenames=6, mincitenames=1]{biblatex}
\addbibresource{thesisbib.bib}



 \renewbibmacro{in:}{}
    \DeclareFieldFormat[article]{volume}{\emph{#1}}%
%    \DeclareFieldFormat[article]{number}{(#1)}%    
\DeclareFieldFormat{pages}{#1}

%    \renewcommand{\multinamedelim}{, }%
    \renewcommand{\finalnamedelim}{\ \&\  }%

\renewbibmacro*{volume+number+eid}{%
  \printfield{volume}%
%  \setunit*{\adddot}% DELETED
%  \setunit*{\addnbspace}
  % NEW (optional); there's also \addnbthinspace
  \printfield{number}%
  \setunit{\addcomma\space}%
  \printfield{eid}}
\DeclareFieldFormat[article]{number}{\mkbibparens{#1}}

\AtEveryBibitem{%
   \DeclareFieldFormat[article,inproceedings]{title}{#1}%
}



\AtEveryCitekey{%
  \ifkeyword{kobib}{%
    \renewcommand{\multinamedelim}{, }%
    \renewcommand{\finalnamedelim}{, }%
%    \DeclareDelimFormat[Book]{andothersdelim}{asdfadsf}
%\DefineBibliographyStrings{english}{andothers={\&~al\adddot}}
  }{%
  }%
}

\DefineBibliographyStrings{english}{phdthesis = {박사학위논문}}
\DefineBibliographyStrings{english}{mathesis = {석사학위논문}}

\newbibmacro*{institution+thesistype+location+date}{%
  \printlist{location}%
  \iflistundef{institution}
    {\setunit*{\addcomma\space}}
    {\setunit*{\addcolon\space}}%
  \printlist{institution}%
  \setunit*{\addspace}%
  \printfield{type}%
  \setunit*{\addcomma\space}%
  \usebibmacro{date}%
  \newunit}

\usepackage{xpatch}
\xpatchbibdriver{thesis}
  {\printfield{type}%
   \newunit
   \usebibmacro{institution+location+date}}
  {\usebibmacro{institution+thesistype+location+date}}
  {}{}
  
  \AtEveryBibitem{%
  \ifkeyword{kobib}{%
    \renewcommand{\multinamedelim}{, }%
    \renewcommand{\finalnamedelim}{, }%
    
   \DeclareFieldFormat[article]{volume}{\textbf{#1}}%
    \DeclareFieldFormat*{journaltitle}{#1,}
\DeclareFieldFormat*{titlecase}{%
    \ifdef{\currentfield}
      {\ifcurrentfield{journaltitle}
         {\usefield{\textbf}{\currentfield}}%
         {#1}}
      {#1}}
\DeclareFieldFormat*[book, thesis]{title}{#1}
\DeclareFieldFormat*[book, thesis]{titlecase}{%
    \ifdef{\currentfield}
      {\ifcurrentfield{title}
         {\usefield{\textbf}{\currentfield}}%
         {#1}}
      {#1}}
  }{%
  }%
}


\DeclareDelimFormat[textcite]{nameyeardelim}{}
%\DeclareDelimFormat[textcite]{textcitedelim}{}
%%%%%%%%%%%%%%
